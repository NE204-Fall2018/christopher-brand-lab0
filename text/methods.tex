The energy calibration was performed using a two-point linear fit between
$^{137}$Cs and $^{241}$Am. To perform the calibration, a program searched
the raw spectrum data of $^{137}$Cs and $^{241}$Am looking for
the largest peaks within the spectrum. The program iterated over the spectrum for the
number of gamma-ray energies present since there should only be peaks corresponding
to the number of gamma-ray energies present. With a peak found, the centroid of the peak was
recorded, and subsequently, utilizing a pre-defined width that incapsulates
the whole peak, the program fit a Gaussian model and
a linear model to this portion of the data. After modeling the data, the peak was
set to zero so during the next iteration the same peak is not found again.
The width of the peak should be used carefully. A peak of interest could be deleted on accident.

This approximation works for HPGe since the peaks produced have sharp resolution.
