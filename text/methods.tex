\subsection{Experimental Setup}
A countrate energy spectrum dataset consisting of  five sources wwas gathered with a coaxial high purity germanium detector (HPGE) by instructor Ross Barnowskifor. Out of the five sources $^{241}$Am and $^{137}$Cs were used to calibrate the energy spectrum and the peaks in $^{133}$Ba was used to benchmark the accuracy of the calibration. The known energies of the calibration peaks are presented below in Table~\ref{tab:CalSrc}.

\begin{table}[H]
        \begin{center}
                \begin{tabular}{l|r}
                        \textbf{Source} & \textbf{Energy (keV)}\\
                        \hline
                        $^{241}$Am      &       59.5    \\
                        $^{137}$Cs      &       661.6   \\
                \end{tabular}
                \caption{Source Isotopes and Corresponding Gamma-ray energies}
                \label{tab:CalSrc}
        \end{center}
\end{table}

\subsection{Peak Selection}
	
	A program for peak selection was created using python 3.6, after inspection of the spectrum. The program iterated over the spectrum assuming a set number of gamma-ray energies were present because the spectrums consisted of only one source. With this knowledge, the max value was determined and the peak selected with a preset bin width. The peak information was recorded and then deleted to find the next max/peak.  

\subsection{Gaussian and Linear Fit}

        The gamma-ray spectrum is approximated as being composed of a global non-linear background with guassian peaks due to the impingment of high branching ratio gamma-ray lines from radioactive decay.\cite{Knoll} The peaks are fitted by a Gaussian of the form:

\begin{equation}
G(x; A,\mu, \sigma) = A\exp\bigg(-\frac{(x-\mu)^2}{2\sigma^2}\bigg)
\end{equation}

Additionally, the data was fit to a linear model and the two were summed together to achieve the final result. 



